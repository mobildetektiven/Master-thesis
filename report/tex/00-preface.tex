\chapter*{Preface}
\addcontentsline{toc}{chapter}{Preface}

    
    There are four listed supervisors for this thesis. Lars Imsland has been my main supervisor, we have met almost every other week during the thesis. Lars has been available for discussions about the case and methods used in this thesis. Anders Willersrud has been my co-supervisor from Hymatek Controls. Anders has been available for discussion during the thesis as well. Mostly through Skype, but I was at one occasion in Oslo at Hymatek to discuss the different possible cases I had found in the dataset. Beatriz Galindo-Prieto and Frank Westad were brought on-board as co-supervisors to help with missing data in dataset. Both Beatriz and Frank looked at the data to try to solve the problem with missing data, in addition we discussed possible techniques for feature selection from a larger dataset. Unfortunately no good solutions were found to the missing data problem, and my work with Frank and Beatriz did not result in any contribution to the thesis.
    
    During this thesis I have been working on a dataset from 27 Hydroelectric power plants provided by Hymatek. In addition they provided a log of recorded incidents and maintenance at the plants. I have been working on a Dell desktop provided by NTNU in Ubuntu, using the Anaconda distribution to manage my Python environment. Keras and Scikit-learn have been used as machine learning libraries. In my project thesis I looked into the possibility to classify the condition of the guide vanes of a Francis turbine. The code used for one class support vector machines is reused from my project thesis. The rest of the framework used in this thesis, has been written by me during the master thesis. This includes a library for reading and converting the dataset provided by Hymatek into a form that could be analyzed. 
    
    In this thesis three different anomaly detection techniques are tested on Pelton turbines. One class support vector machine was used in my project thesis. The two others are proposed by myself after searching the literature. I found the Pelton case when looking through the data and plant logs provided by Hymatek. It was chosen as a good use-case in cooperation with Hymatek and Lars. No literature has been provided, and all papers and articles used in this thesis are a result of my own literary research.

    I want to thank all my supervisors for their help during my thesis, and Hymatek for providing such an extensive data set. 
    
    \begin{center}
    Trondheim, \today, \\
    % \vspace{1.5cm}
    Asgeir Øen Åsnes
    \end{center}

    % For at sensorene skal få et klarest mulig inntrykk av bidraget fra kandidaten må dette gå klart fram av masteroppgaven.
    % Studenten må derfor beskrive følgende i forordet til masteroppgaven:
    %     Hvilken informasjon, programvare, utstyr osv. som er stilt til rådighet eller danner basis for arbeidet
    %     Hvilken hjelp som er mottatt og fra hvem underveis
    % Faglærer deltar ikke i sensuren, men må attestere at informasjonen i forordet er riktig
    %     Send derfor forordet til faglærer i god tid før innlevering så dere blir enige om beskrivelsen
    % Innleveringsskjema må leveres umiddelbart etter innlevering av masteroppgaven. Dette er en forutsetning for start av sensur
