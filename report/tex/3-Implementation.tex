%!TEX root = ../Thesis.tex
\chapter{Implementation}\label{cha:implementation}
This chapter introduces the system implementation and the software packages used during the thesis.

\section{Python}
    All code is developed in Python 3.6 through Anaconda and Spyder. Anaconda is a python distribution with focus on data science and machine learning. Anaconda simplifies installation and handling of packages and package dependencies. In addition, it allows for exporting of the package environment, for automatic setup at new locations. Spyder is a Python IDE that has a similar layout to Matlab, where objects and variables can be accessed through a workspace. This makes it a good alternative for data science. 
    
    A library for reading and preparing the data from the energy company is created from scratch. The library is meant to be used for further analysis at Hymatek. The code used for OC SVM is mostly reused from the project thesis. All other code is written during the master thesis. All methods mentioned in theory is implemented and ready for use. Libraries are created that makes it easy to analyze the data on the different methods. 
    
    
\section{Scikit-learn}
Scikit-learn, \cite{scikit-learn} and \cite{scikit-web} is an open source toolbox available for Python. It includes many different machine learning algorithms and the tools needed to preprocess and analyze data. The framework created by scikit-learn is used throughout this thesis, as it includes implementations of OC SVM and KDE. It also has implementations of many other machine learning algorithms, if one wants to expand the analysis with new methods. In addition to the being a great machine learning library for Python, the website \footnote{\url{http://scikit-learn.org/stable/ }}also explains the main theory behind the different algorithms, and what the scikit-learn algorithms are built upon in an easy to find manner. This makes it a good choice when one want to understand what is making the algorithms work.        

\section{Keras}
    Scikit-learn does not include neural networks. Different libraries were considered before Keras was chosen for the NN analysis. Keras \cite{chollet2015keras} is a Python library that simplifies creation of NNs in Python. Instead of having its own implementation of NNs, it uses other NN python libraries as backends. This means that you can run the analysis using, for example, Google's TensorFlow through Keras. At this moment Keras supports Google's TensorFlow, Theano developed by LISA Lab at Université de Montréal and CNTK developed by Microsoft. This means that by using Keras, only one line of code needs to be changed if one want to change between one of the options above. Developing code using one of the NN libraries directly would make switching between the different options more cumbersome. However, one important factor to keep in mind, is that Keras is not officially supported by any of the backend developers. This means that  not all of the original functionality is guaranteed to be supported in Keras. Regardless, the fact that Keras is easy to use and its focus on enabling fast experimentation, are the main reasons for developing in Keras. Keras supports the many different types of NN, including the LSTM RNN used in this thesis. Keras also enables the user to choose to run on either CPU or GPU, making it flexible and easy to optimize to the available hardware. More information about Keras can be found at \footnote{\url{https://keras.io/}}.
        
    \subsubsection{Google TensorFlow}
        For this thesis, only the TensorFlow backend is used. TensorFlow \cite{Abadi} is developed by The Google Brain project, and it started as the DistBelif project back in 2011. Tensor flow is the second generation, built upon the experiences from DistBelif. Tensorflow has support for parallelism and can have many different devices collaborating on the same problem. As stated in \cite{Abadi}, "a computation expressed using TensorFlow can be executed with little or no change on a wide variety of heterogeneous systems, ranging from mobile devices such as phones and tablets up to large-scale distributed systems of hundreds of machines and thousands of computational devices such as GPU cards." This shows the flexibility TensorFlow introduces. More information about TensorFlow can be found at \footnote{\url{https://www.tensorflow.org/}}.
    
    
    