%!TEX root = ../Thesis.tex
\chapter{Conclusion and future work}\label{cha:conclusions}
%

\section{Conclusion}
    % Shit worked! 
    
    It this thesis anomaly detection for hydroelectric power plants have been investigated, attempting to detect anomalies related to the operation of the needles of Pelton turbines. Three different methods are used for the anomaly detection, one class support vector machines, kernel density estimation and long short term memory neural networks The first is a classifier and the two latter are regressors. The methods are trained on three different training sets, and evaluated on four different test cases. All methods showed promise as potential anomaly detectors, where kernel density estimation and long short term memory recurrent neural network performed best. Kernel density estimation is found superior, as it performs almost as well as the neural network which is a much more complicated method. Kernel density estimation has fewer hyperparameters, making it both faster and more straightforward to optimize. In addition it was shown to be the method that generalized best to new data. One class support vector machine was found to have hyperparameters that are very dependent on the training data, and hence the hyperparameters needs to be tuned for every new training set. This makes the method more cumbersome to use than the two others. Kernel density estimation performed very well with the same parameterization for all training sets and test cases. The neural network also performed well, as long as early stopping was enabled. Early stopping avoids overfitting on the training data. The techniques showed promise when testing a pre-trained anomaly detector at data from a new plant. As long as the data transformation used during training was used on data from the new plant, all methods showed that they were able to detect anomalies.           
    
    The challenges of working real industrial data became apparent during this thesis. The data was found to be sampled aperiodically and often with only one process variable at a time. This made building a case for anomaly detection more difficult than expected, and it resulted in only being able to include two process variables in the analysis. Ideally, many more variables would have been sampled, so that more plant information could have been included in the analysis. This could have created a more robust anomaly detection system, and the feature selection and dimensionality reduction techniques introduced in theory could have been tested. 
    
    The two regressors output continuous values and do not classify the data as normal or anomalous. A numeric value evaluating the 0.1\% percent most extreme data was used as a threshold in the analysis. Before the two methods can be used as anomaly detectors a new way to set the threshold must be found. It is suggested that using both a threshold for magnitude and rate of change to give plant operators more information about the condition of the needles, but no exact solution is presented. 
    
    The results showed that system degradation could be detected without having data representing system degradation during training. As sampling data from system failures is tough to come by, this problem is often solved by creating artificial data that represents known failure modes. This does, however, introduce the risk of not detecting unknown failures, and is a non-trivial task that requires a lot of system expertise. By using methods that only requires data from normal system operation, this problem is avoided, and all data used during training can be sampled at the plant. 
    
    
    
\section{Future work}
    A natural next step would be to investigate how to find reasonable thresholds for the regressors. Having this in place would then make the methods ready for testing at a plant. More effort can also be spent on finding the optimal hyperparameters for the neural network, as there might exist parameterizations that yield better performance. Another natural next step would be to collect more data in a way that allows one to perform a similar analysis with more process variables from the plant.

    How to best present the new information to plant operators should also be investigated. It is vital that the system is easy to interpret, and that the operators trust it. The work from this thesis and the work from \cite{Aasnes2017} will be a major part of a paper written for HYDRO 2018 Progress through Partnerships that will be presented in Gdansk, Poland 15 - 17 October 2018. 

    % get data with even sampling rate to test performance with more variables available
    % talk to energy companies, how would they want such a system to behave?
    % look into how to visualise the data 
    % feature selection methods once one get samples that are even 
    % try to combine the work from the project and master thesis to see how one can create a system more independent of the plant setup. 
    
    % Data from this plant could be used to generalize the anomaly detectors to any combination of needles, but this will not be covered in this thesis. 
    
    % writing a paper for conference based on project and master thesis. 
    
    % \subsection{Implementation at plants}
       