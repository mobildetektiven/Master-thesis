%!TEX root = ../Thesis.tex
\chapter{Implementation}\label{cha:contributions}
This chapter introduces the software packages used during the thesis. Some of software developed for the thesis is constructed on a framework built during my project thesis. Most algorithms are used preimplemented from different sources.   

\section{Scikit-learn}
Scikit-learn \cite{scikit-learn}, \cite{scikit-web} is an open source toolbox available for Python. It includes a number of different machine learning algorithms and the tools needed to preprocess and analyze your data. The framework created by scikit-learn is used throughout this project, mainly because it either includes the algorithms that are to be tested, or there exist an API that allows you to run the non existing algorithms through the scikit-learn framework. 

In addition to the being a great machine learning library for Python, the website \footnote{\url{http://scikit-learn.org/stable/ }}also explains the main theory behind the different algorithms, and what the scikit-learn algorithms are built upon in a easy to find manner. This makes it a good choice when one want to understand what is making the algorithms work.        

\section{Keras}
    Scikit-learn does not include neural networks. Different libraries were considered before Keras was chosen for the NN analysis. Keras \cite{chollet2015keras} is a Python library that simplifies creation of NNs in Python. Instead of having its own implementation of NNs, it uses other NN python libraries as backends. This means that you can run you analyzis using for example Google's TensorFlow through Keras. At this moment Keras supports Google's TensorFlow, Theano developed by LISA Lab at Université de Montréal and CNTK developed by Microsoft. This means that by using Keras, only one line of code needs to be changed if one want to change between one of the options above. Developing code using one of the NN libraries directly would make switching between the different options more cumbersome. However, one important factor to keep in mind, is that Keras is not officially supported by any of the backend developers. This means that  not all of the original functionality is guaranteed to be supported in Keras. Regardless, the fact that Keras is easy to use and its focus on enabling fast experimentation, are the main reasons for developing in Keras. Keras supports the normal sequential NN which is used in this analysis, but it also has support for the more complex convolutional and reccurent NN. It can also run on both CPU and GPU, making it flexible and optimizable to the hardware it runs on. More information about Keras can be found at \footnote{\url{https://keras.io/}}.
        
    \subsubsection{Google TensorFlow}
        For this project only the TensorFlow backend is tested. TensorFlow \cite{Abadi} is developed by The Google Brain project, and it started as the DistBelif project back in 2011. Tensor flow is the second generation, built upon the experiences from DistBelif. Tensorflow has support for parallelism and has the possibility to have many different devices collaborating on the same problem. As stated in \cite{Abadi}, "a computation expressed using TensorFlow can be executed with little or no change on a wide variety of heterogeneous systems, ranging from mobile devices such as phones and tablets up to large-scale distributed systems of hundreds of machines and thousands of computational devices such as GPU cards". This shows the flexibilty TensorFlow introduces. More information about TensorFlow can be found at \footnote{\url{https://www.tensorflow.org/}}.
    
    
    The \textbf{Pandas} package \cite{Mckinney2010} provides efficient and intuitive methods and data structures that serves as a great basis for data analysis. 